\documentclass{article}
\usepackage{amsmath}
\usepackage[utf8]{inputenc}

\newcounter{problem}
\newcounter{solution}

\renewcommand{\abstractname}{Logique et Algorithmie - 2014} 

\newcommand\Problem{%
  \stepcounter{problem}%
  \textbf{\theproblem.}~%
  \setcounter{solution}{0}%
}

\newcommand\TheSolution{%
  \textbf{Solution:}\\%
}

\newcommand\Hint{%
  \textbf{Hint:}~%
}

\newcommand\ASolution{%
  \stepcounter{solution}%
  \textbf{Solution \thesolution:}\\%
}
\parindent 0in
\parskip 1em

\begin{document}

\begin{abstract}
Voilà une petite série d'exercices pour vous entraîner sur les notions de logiques et d'algorithmie que nous avons vu lors des deux premiers cours. À titre indicatif : ce document a été rédigé à l'aide du langage de formatage \TeX\space  inventé par Donald Knuth lorsqu'il rédigeait \textit{The Art Of Computer Programming} en 1977 !
\end{abstract}

\textbf{Logique :}\\

\Problem Écrivez les tables de vérité des propositions suivantes :

\Hint Pour être sûr de ne pas en oublier : il y a \(2^n\) lignes lorsqu'il y a \(n\) variables booléennes différentes. Par exemple si les variables sont \(P\), \(Q\) et \(R\) alors il y aura \(2^3 = 8\) lignes dans notre table de vérité.

\Hint On rappelle également que \(\land\) signifie \textbf{et}, que \(\lor\) signifie \textbf{ou} et que \(\neg\) signifie \textbf{non}.

\[ (P \land Q) \lor R \]
\[ (P \lor Q) \land R \]
\[ (P \lor Q) \land \neg P \]

\Problem Parmis les propositions suivantes, lesquelles sont satisfaisables ? Donnez une solution si c'est le cas ou justifiez (à l'aide de mots ou d'une table de vérité) dans le cas contraire.

\Hint On rappelle qu'une proposition est satisfaisable si on peut trouver une configuration des variables (ie : une ligne dans la table de vérité) telle que la proposition est vraie. Il peut y en avoir plusieurs.

\[ (P \land \neg P) \land Q \]
\[ (P \land \neg P) \lor Q \]
\[ (P \land \neg Q) \lor Q \]
\[ (P \land \neg Q) \lor (\neg P \land Q) \]

\Hint Pour les proposition suivante, n'essayez pas d'écrire la table de vérité, il vous faudrait écrire jusqu'à \(2^{26} = 67108864\) lignes !

\[ A \land \neg B \land C \land \neg D \dots \land \neg X \land Y \land \neg Z \]
\[ A \land \neg (B \lor C \dots \lor Z) \]
\[ A \land \neg (A \lor B \lor C \dots \lor Z) \]


\Problem Simplifiez les conditions suivantes :

\Hint Choisissez astucieusement vos variables booléennes (P, Q, etc.) pour traduire en langage logique

\[ ( x > 0 \text{ or } ( x \leq 0 \text{ and } y > 100 )) \]
\[ \text{ not } (x \neq 0 \text{ or } y \neq 0) \]
\[ \text{ not } (x \neq 0 \text{ or } y \leq 0) \]

\textbf{Algorithmie :} On se propose d'étudier le code suivant qui calcule le \(n\)ième nombre de la suite de Fibonnaci:

\begin{verbatim}
Entrée : n
Début :
  a <- 0
  b <- 1
  tant que n > 0
    t <- a+b
    a <- b
    n <- n-1
  Retour : b
\end{verbatim}

\Problem Écrivez pas à pas l'exécution de cette fonction pour \(n=8\)

\begin{table}[h]
\begin{tabular}{lllll}
 \text{step} & 1  & 2 & 3 & \dots  \\ \hline
 \verb|n| &  &  &  &  \\ \hline
 \verb|a| &  &  &  &  \\ \hline
 \verb|b| &  &  &  &  \\ \hline
\end{tabular}
\end{table}

\Problem Que se passe-t-il si \(n=0\) ?

\Problem Montrez que cette fonction est un algorithme

\Problem Écrivez cette fonction en PHP. Testez là avec des valeurs de plus en plus grande de \(n\). Comptez le nombre d'itération effectuées. Tracez le graphe de sa complexité.

\Hint N'oubliez pas que le graphe de la complexité se trace en fonction de la \textit{taille} de \(n\) et non pas de sa valeur !

\end{document}